Question 1
Using the NetLogo simulation of a diffusion process on a small world topology, answer the following. If the 
infection rate is 0.25, and the recovery rate is 0.30, what is true of the diffusion processes in the regular
lattice vs. rewired case:
shortcuts allow all nodes to be simultaneously in the infected state
shortcuts prolong the amount of time an established infection can stay in the network
a pure lattice topology (no shortcuts) allows the infection to become established more quickly
Question Explanation

Look for the number of infected individuals in the long run. You may need to reinfect the network a 
few times to start to see the difference in behavior.
%========================================================================%
Question 2
Using the NetLogo model of graph coloring on a network grown randomly or preferentially, answer the following
(setting m = 1). What is true about a time it takes for the network to find a solution:
preferential attachment generates a topology that is solved more quickly than one generated with random attachment
preferential attachment generates a topology that is solved more slowly than one generated with random attachment.
the average time to solution is unaffected by whether the growth is random or preferential
Question Explanation

If you're not sure about the answer here (though VARY-BA-TOPOLOGY should allow you to answer this), 
check Kearns et al. paper (listed in the syllabus) where they used human subjects to run this experiment and generally
get similar results.
%========================================================================%
Question 3
Use the NetLogo cascade model with the 19_4 network (setup19_4 button, a = 3, b = 2, bilingual = off) and allocate 
opinion at random (or you can manually set the nodes' opinions using the select-blue and select-red buttons and 
clicking on individual nodes). Then allow the nodes to update their opinions until everyone is settled into 
their opinions. With these payoffs, how many distinct communities do you observe that can have a separate opinion from neighboring communities?

%------------------%
Question Explanation

With these payoffs you should be able to observe different opinions persisting 
in different parts of the network. Try setting individual nodes or re-randomizing to see a range of behaviors
depending on the initial allocation of choices.
%========================================================================%
Question 4
Using the same NetLogo cascade model with the 19_4 network (setup19_4 button, b = 2, bilingual = off), how high does a, the payoff for choosing blue at the same time as a friend, need to be such that the whole network will adopt blue every time if at least one node adopts blue?
5
3
7
1
Question Explanation

To figure this one out, you may want to alloc-opinion with init-prob-blue 0. That way the whole network 
will have initially chosen red. Then use select-blue to set just a single node's opinion to blue.
%========================================================================%
Question 5
Use the NetLogo model of innovation on a network to answer the following. Relative to the average maximum
solution achieved on a randomly grown topology, a network grown with preferential attachment will
have roughly the same time to solution and max fitness
at least 10% lower max final fitness and 50% faster convergence to solution
at least 10% higher max final fitness and 50% faster convergence to solution
Question Explanation

Run the model repeatedly at the two extremes (prob-pref = 0 and prob-pref = 1) and note how quickly the model converges and what is shown under agent-max.
