Social network analysis (SNA) is the use of network theory to analyse social networks. Social network analysis views social relationships in terms of network theory, consisting of nodes, representing individual actors within the network, and ties which represent relationships between the individuals, such as friendship, kinship, organizations and sexual relationships.[1][2] These networks are often depicted in a social network diagram, where nodes are represented as points and ties are represented as lines.
Social network analysis has emerged as a key technique in modern sociology. It has also gained a significant following in anthropology, biology, communication studies, economics, geography, history, information science, organizational studies, political science, social psychology, development studies, and sociolinguistics and is now commonly available as a consumer tool.[3][4][5][6]
