The critical path method (CPM) is a way of breaking complex projects into lists of activities, and determining which are critical to keeping a project on schedule. Developed in the 1950s, this method was used for complicated government and private industry programs that were running behind schedule for undetermined reasons. A critical path as defined by the method as a series of events, called activities, which must be completed in the correct order and on time.


%==================================================================================================%
%-https://www.thebalance.com/critical-path-description-and-overview-2276120
\section{Critical Path}


Critical Path is a term from the field of project management describing a set of tools and a methodology.

The technical definition of the critical path in a sequence of networked work packages is the path with the least amount of slack. In practical terms, this path is the sequence of events that if any are delayed, will delay the entire project. And in even simpler terms, the critical path is the sequence of tasks that will take the longest to complete to deliver the project.


%====================================================================================================% 

Critical Path Project Management describes a methodology for managing and controlling a project using a distinct set of tools. 

\subsection{Identification of the Critical Path:}
Project managers work with team members to define all of the work required to complete a project or to achieve the project scope. The work is typically broken down into units called work packages. These work packages are small enough in size that they can be associated with an owner, managed for risk and controlled for time, cost and materials.

Work packages that are too small are meaningless, and work packages that are too large provide no basis for effective management and control. A commonly described benchmark is that no work package should take less than eight hours to complete or more than eighty hours. It, of course, may vary depending upon the nature of the project. 

Each work team defines and estimates the duration and cost for their work packages.

The project manager pools the work packages and sequences them in the order in which they must be completed. The estimates of duration and the sequence of events are used to construct a network diagram, where several key measures are identified. These include: 

Early Start: the earliest a work package can start. 
Early Finish: the earliest a work package can be completed.
Late Start: the latest a work package can be started and not delay the project.
Late Finish: the latest a work package can be finished and not delay the project.
Slack (Float): the amount of time a work package can be delayed and not impact the project.
These metrics are used to calculate the various paths through the network, producing a duration and identifying available slack for individual work packages. The project manager and project team members typically adjust the sequencing of the events and look at different options. Once resources are defined the work packages are sequenced yet again, looking for the most efficient, timely and least risky project plan. 

The project team can view the actual paths through the network, and the one (or more) that has the least amount of available slack (sometimes zero), meaning if any item on that path is delayed even by one day, the project will be delayed correspondingly. 

It is important to note the following:

There can be more than one critical path in a network diagram.
The critical path can change based on resource scheduling. 
The critical path can change during the execution of a project. 

%---------------------------------------%
\subsection{Uses of the Critical Path:}

The critical path is an important project management tool that allows the project manager and team to focus their efforts on the most important work packages. Common actions focused on the critical path include: 

\begin{itemize}
\item Ensuring that resources are available when needed 
\item Borrowing resources from non-critical work packages to ensure completion of the critical items.
\item Careful monitoring and reporting for critical activities.
\item Smoothing or leveling of resources to most efficiently complete tasks on the critical path.
\item Crashing the schedule by adding resources to those items where the most cost-effective reduction in duration can be achieved. 
\end{itemize}
%---------------------------------------%
Identification of work that can be completed simultaneously (fast tracking) to ensure that there are no delays along the critical path. 
Use of the network diagram to assess the overall risk of the project. If the network has multiple critical paths or one critical path and several near-critical paths, it is said to be sensitive. The more sensitive a project network, the larger the risk of delays and the harder the job of the project manager and team for monitoring task completion. 

\newpage

%==============================================================%

Identifying the Critical Path and Critical Tasks
%- https://www.sqa.org.uk/e-learning/ProjMan02CD/page_11.htm

Critical path analysis is an important element of project planning. The critical path is the longest-duration path through the network. The tasks that lie on the critical path cannot be delayed without delaying the project. The critical path can be identified by determining the following parameters for each task:

Earliest Start Time (ES): the earliest time at which the task can start, given that any predecessor tasks must be completed first.
Earliest Finish Time (EF): the earliest start time for the task plus the time required to complete the task.
Latest Finish Time (LF): the latest time at which the task can be completed without delaying the project.
Latest Start Time (LS): the latest finish time minus the time required to complete the task.
The slack time or float for a task is the time between its earliest and latest start time, or between its earliest and latest finish time, or, to put it another way, slack is the amount of time that a task can be delayed past its earliest start or earliest finish without delaying the project. If the earliest and latest end times are the same, the task is critical.

The critical path is the path through the network in which none of the tasks have slack, that is, the path for which ES=LS and EF=LF for all tasks in the path. Any delay in the critical path delays the whole project. Note that if we wish to finish the project sooner, it is necessary to reduce the total time required for the activities in the critical path.

%====================================================================%
\section{PERT}
%- http://www.wisegeek.com/what-is-the-program-evaluation-and-review-technique.htm

The program evaluation and review technique (PERT) is a project management system that resembles a dynamic flowchart of interrelated processes. It is used to coordinate diversified project elements and their respective influences on costs, time and each other. This technique provides a more adaptive overview of these dynamic elements than traditional static project charts and timelines. Originally developed for large scale military-industrial projects, the program evaluation and review technique is employed in large and small-scale organizations that require the coordination of resources, teams, costs and deadlines in order to achieve dedicated outcomes.

A PERT chart gives an overview of a project development process. In practice, the execution of tasks depends on ongoing project requirements, team decisions and other external constraints. The primary task when developing a PERT plan is determining the critical activities upon which all other activities depend. This is sometimes referred to as the critical path method (CPM).

The chart itself consists of three major elements — nodes, arrows and paths — put together in various tree formations. The nodes identify the project's key elements, such as a departmental review, a research and development trial or the public launch of a new product. These nodes are connected with arrows, and the arrows determine the sequence of stages through which a project will go. Some nodes will have multiple arrows dependent upon their outcomes, with either/or decisions or possibly simultaneous activities. The advantage is that the observer can see immediately which project elements will be influenced directly by a given node's processes.

The path of a project is not necessarily linear or static, as one might find in the neat vertical bars of a Gantt chart. A chain is only as strong as its weakest link, so a chart utilizing the program evaluation and review technique is only as timely as its longest path. This path represents the minimum timeframe for completion of a project; as such it is called the critical path. This path will be the area of most concern for a project manager dealing with unforeseen delays and unexpected costs.

In general, the PERT chart displays sequences of processes that might occur simultaneously or might depend on the completion of a previous task. By clear assessment of these nodes, a manager can better recognize potential problem areas and processes most likely to introduce “slippage” into even highly complex projects. By comparing paths of lesser importance to the critical path, managers can identify not only definite milestones and deadlines but also areas of "slack," which afford more wiggle room.

By forcing project managers to establish firm understanding of the critical elements in a project and oversee dependency relationships between nodes, the program evaluation and review technique provides for a clearer grasp of a complex project. It might result in greater team coordination, more efficient communication activities and more effective process or goal evaluation. Computers assist in analysis of more complex plans. Given clear start/finish times and goals, they process known elements using algorithms to output the most thorough forecast possible.

Given the complexity and dynamic nature of PERT planning, the best results might occur with proven industrial methods where processes and expectations are widely known. Unforeseen circumstances or difficulties, scope creep and butterfly effects cans sometimes make short work of any well-laid plan, however. A well-prepared project manager can use the program evaluation and review technique not only to coordinate a vast amount of elements but also to minimize the likeliest areas of risk in order to produce a successful project outcome.

%=====================================================================================================%
\newpage
\section{The Critical Path Method (CPM)}
%- http://www.project-management-skills.com/critical-path-method.html
The Critical Path Method (CPM) can help you keep your projects on track.

Critical path schedules...

Help you identify the activities that must be completed on time in order to complete the whole project on time.
Show you which tasks can be delayed and for how long without impacting the overall project schedule.
Calculate the minimum amount of time it will take to complete the project.
Tell you the earliest and latest dates each activity can start on in order to maintain the schedule.
Recommended Resource...
A Basic Guide to Activity-On-Node and Critical Path Method

The CPM has four key elements...
\begin{enumerate}
\item Critical Path Analysis
\item Float Determination
\item Early Start & Early Finish Calculation
\item Late Start & Late Finish Calculation
\end{enumerate}

Critical Path Analysis

The critical path is the sequence of activities with the longest duration. A delay in any of these activities will result in a delay for the whole project. Below are some critical path examples to help you understand the key elements...

Critical Path Method (CPM)
Using the Critical Path Method (CPM)
The duration of each activity is listed above each node in the diagram. For each path, add the duration of each node to determine it's total duration. The critical path is the one with the longest duration.

There are three paths through this project...

Critical Path Analysis
Use Critical Path Analysis to find Your Critical Path
Float Determination

Once you've identified the critical path for the project, you can determine the float for each activity. Float is the amount of time an activity can slip before it causes your project to be delayed. Float is sometimes referred to as slack.

Figuring out the float using the Critical Path Method is fairly easy. You will start with the activities on the critical path. Each of those activities has a float of zero. If any of those activities slips, the project will be delayed.

Then you take the next longest path. Subtract it's duration from the duration of the critical path. That's the float for each of the activities on that path.

You will continue doing the same for each subsequent longest path until each activities float has been determined. If an activity is on two paths, it's float will be based on the longer path that it belongs to.

Critical Path Float )
Determining Float
Using the critical path diagram from the previous section, Activities 2, 3, and 4 are on the critical path so they have a float of zero.

The next longest path is Activities 1, 3, and 4. Since Activities 3 and 4 are also on the critical path, their float will remain as zero. For any remaining activities, in this case Activity 1, the float will be the duration of the critical path minus the duration of this path. 14 - 12 = 2. So Activity 1 has a float of 2.

The next longest path is Activities 2 and 5. Activity 2 is on the critical path so it will have a float of zero. Activity 5 has a float of 14 - 9, which is 5. So as long as Activity 5 doesn't slip more than 5 days, it won't cause a delay to the project.

\subsection{Early Start & Early Finish Calculation}

The Critical Path Method includes a technique called the Forward Pass which is used to determine the earliest date an activity can start and the earliest date it can finish. These dates are valid as long as all prior activities in that path started on their earliest start date and didn't slip.

Starting with the critical path, the Early Start (ES) of the first activity is one. The Early Finish (EF) of an activity is its ES plus its duration minus one. Using our earlier example, Activity 2 is the first activity on the critical path: ES = 1, EF = 1 + 5 -1 = 5.

Critical Path Schedules
Critical Path Schedules
You then move to the next activity in the path, in this case Activity 3. Its ES is the previous activity's EF + 1. Activity 3 ES = 5 + 1 = 6. Its EF is calculated the same as before: EF = 6 + 7 - 1 = 12.

If an activity has more than one predecessor, to calculate its ES you will use the activity with the latest EF.

\subsection{Late Start & Late Finish Calculation}

The Backward Pass is a Critical Path Method techique you can use to determine the latest date an activity can start and the latest date it can finish before it delays the project.

You'll start once again with the critical path, but this time you'l begin from the last activity in the path. The Late Finish (LF) for the last activity in every path is the same as the last activity's EF in the critical path. The Late Start (LS) is the LF - duration + 1.

In our example, Activity 4 is the last activity on the critical path. Its LF is the same as its EF, which is 14. To calculate the LS, subtract its duration from its LF and add one. LS = 14 - 2 + 1 = 13.

You then move on to the next activity in the path. Its LF is determined by subtracting one from the previous activity's LS. In our example, the next Activity in the critical path is Activity 3. Its LF is equal to Activity 4 LS - 1. Activity 3 LF = 13 -1 = 12. It's LS is calculated the same as before by subtracting its duration from the LF and adding one. Activity 3 LS = 12 - 7 + 1 = 6.

You will continue in this manner moving along each path filling in LF and LS for activities that don't have it already filled in.
